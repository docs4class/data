% Options for packages loaded elsewhere
\PassOptionsToPackage{unicode}{hyperref}
\PassOptionsToPackage{hyphens}{url}
%
\documentclass[
]{book}
\usepackage{amsmath,amssymb}
\usepackage{lmodern}
\usepackage{iftex}
\ifPDFTeX
  \usepackage[T1]{fontenc}
  \usepackage[utf8]{inputenc}
  \usepackage{textcomp} % provide euro and other symbols
\else % if luatex or xetex
  \usepackage{unicode-math}
  \defaultfontfeatures{Scale=MatchLowercase}
  \defaultfontfeatures[\rmfamily]{Ligatures=TeX,Scale=1}
\fi
% Use upquote if available, for straight quotes in verbatim environments
\IfFileExists{upquote.sty}{\usepackage{upquote}}{}
\IfFileExists{microtype.sty}{% use microtype if available
  \usepackage[]{microtype}
  \UseMicrotypeSet[protrusion]{basicmath} % disable protrusion for tt fonts
}{}
\makeatletter
\@ifundefined{KOMAClassName}{% if non-KOMA class
  \IfFileExists{parskip.sty}{%
    \usepackage{parskip}
  }{% else
    \setlength{\parindent}{0pt}
    \setlength{\parskip}{6pt plus 2pt minus 1pt}}
}{% if KOMA class
  \KOMAoptions{parskip=half}}
\makeatother
\usepackage{xcolor}
\usepackage{color}
\usepackage{fancyvrb}
\newcommand{\VerbBar}{|}
\newcommand{\VERB}{\Verb[commandchars=\\\{\}]}
\DefineVerbatimEnvironment{Highlighting}{Verbatim}{commandchars=\\\{\}}
% Add ',fontsize=\small' for more characters per line
\usepackage{framed}
\definecolor{shadecolor}{RGB}{248,248,248}
\newenvironment{Shaded}{\begin{snugshade}}{\end{snugshade}}
\newcommand{\AlertTok}[1]{\textcolor[rgb]{0.94,0.16,0.16}{#1}}
\newcommand{\AnnotationTok}[1]{\textcolor[rgb]{0.56,0.35,0.01}{\textbf{\textit{#1}}}}
\newcommand{\AttributeTok}[1]{\textcolor[rgb]{0.77,0.63,0.00}{#1}}
\newcommand{\BaseNTok}[1]{\textcolor[rgb]{0.00,0.00,0.81}{#1}}
\newcommand{\BuiltInTok}[1]{#1}
\newcommand{\CharTok}[1]{\textcolor[rgb]{0.31,0.60,0.02}{#1}}
\newcommand{\CommentTok}[1]{\textcolor[rgb]{0.56,0.35,0.01}{\textit{#1}}}
\newcommand{\CommentVarTok}[1]{\textcolor[rgb]{0.56,0.35,0.01}{\textbf{\textit{#1}}}}
\newcommand{\ConstantTok}[1]{\textcolor[rgb]{0.00,0.00,0.00}{#1}}
\newcommand{\ControlFlowTok}[1]{\textcolor[rgb]{0.13,0.29,0.53}{\textbf{#1}}}
\newcommand{\DataTypeTok}[1]{\textcolor[rgb]{0.13,0.29,0.53}{#1}}
\newcommand{\DecValTok}[1]{\textcolor[rgb]{0.00,0.00,0.81}{#1}}
\newcommand{\DocumentationTok}[1]{\textcolor[rgb]{0.56,0.35,0.01}{\textbf{\textit{#1}}}}
\newcommand{\ErrorTok}[1]{\textcolor[rgb]{0.64,0.00,0.00}{\textbf{#1}}}
\newcommand{\ExtensionTok}[1]{#1}
\newcommand{\FloatTok}[1]{\textcolor[rgb]{0.00,0.00,0.81}{#1}}
\newcommand{\FunctionTok}[1]{\textcolor[rgb]{0.00,0.00,0.00}{#1}}
\newcommand{\ImportTok}[1]{#1}
\newcommand{\InformationTok}[1]{\textcolor[rgb]{0.56,0.35,0.01}{\textbf{\textit{#1}}}}
\newcommand{\KeywordTok}[1]{\textcolor[rgb]{0.13,0.29,0.53}{\textbf{#1}}}
\newcommand{\NormalTok}[1]{#1}
\newcommand{\OperatorTok}[1]{\textcolor[rgb]{0.81,0.36,0.00}{\textbf{#1}}}
\newcommand{\OtherTok}[1]{\textcolor[rgb]{0.56,0.35,0.01}{#1}}
\newcommand{\PreprocessorTok}[1]{\textcolor[rgb]{0.56,0.35,0.01}{\textit{#1}}}
\newcommand{\RegionMarkerTok}[1]{#1}
\newcommand{\SpecialCharTok}[1]{\textcolor[rgb]{0.00,0.00,0.00}{#1}}
\newcommand{\SpecialStringTok}[1]{\textcolor[rgb]{0.31,0.60,0.02}{#1}}
\newcommand{\StringTok}[1]{\textcolor[rgb]{0.31,0.60,0.02}{#1}}
\newcommand{\VariableTok}[1]{\textcolor[rgb]{0.00,0.00,0.00}{#1}}
\newcommand{\VerbatimStringTok}[1]{\textcolor[rgb]{0.31,0.60,0.02}{#1}}
\newcommand{\WarningTok}[1]{\textcolor[rgb]{0.56,0.35,0.01}{\textbf{\textit{#1}}}}
\usepackage{longtable,booktabs,array}
\usepackage{calc} % for calculating minipage widths
% Correct order of tables after \paragraph or \subparagraph
\usepackage{etoolbox}
\makeatletter
\patchcmd\longtable{\par}{\if@noskipsec\mbox{}\fi\par}{}{}
\makeatother
% Allow footnotes in longtable head/foot
\IfFileExists{footnotehyper.sty}{\usepackage{footnotehyper}}{\usepackage{footnote}}
\makesavenoteenv{longtable}
\usepackage{graphicx}
\makeatletter
\def\maxwidth{\ifdim\Gin@nat@width>\linewidth\linewidth\else\Gin@nat@width\fi}
\def\maxheight{\ifdim\Gin@nat@height>\textheight\textheight\else\Gin@nat@height\fi}
\makeatother
% Scale images if necessary, so that they will not overflow the page
% margins by default, and it is still possible to overwrite the defaults
% using explicit options in \includegraphics[width, height, ...]{}
\setkeys{Gin}{width=\maxwidth,height=\maxheight,keepaspectratio}
% Set default figure placement to htbp
\makeatletter
\def\fps@figure{htbp}
\makeatother
\setlength{\emergencystretch}{3em} % prevent overfull lines
\providecommand{\tightlist}{%
  \setlength{\itemsep}{0pt}\setlength{\parskip}{0pt}}
\setcounter{secnumdepth}{5}
\usepackage{booktabs}
\ifLuaTeX
  \usepackage{selnolig}  % disable illegal ligatures
\fi
\usepackage[]{natbib}
\bibliographystyle{apalike}
\IfFileExists{bookmark.sty}{\usepackage{bookmark}}{\usepackage{hyperref}}
\IfFileExists{xurl.sty}{\usepackage{xurl}}{} % add URL line breaks if available
\urlstyle{same} % disable monospaced font for URLs
\hypersetup{
  pdftitle={A Minimal Book Example},
  pdfauthor={John Doe},
  hidelinks,
  pdfcreator={LaTeX via pandoc}}

\title{A Minimal Book Example}
\author{John Doe}
\date{2023-01-06}

\usepackage{amsthm}
\newtheorem{theorem}{Theorem}[chapter]
\newtheorem{lemma}{Lemma}[chapter]
\newtheorem{corollary}{Corollary}[chapter]
\newtheorem{proposition}{Proposition}[chapter]
\newtheorem{conjecture}{Conjecture}[chapter]
\theoremstyle{definition}
\newtheorem{definition}{Definition}[chapter]
\theoremstyle{definition}
\newtheorem{example}{Example}[chapter]
\theoremstyle{definition}
\newtheorem{exercise}{Exercise}[chapter]
\theoremstyle{definition}
\newtheorem{hypothesis}{Hypothesis}[chapter]
\theoremstyle{remark}
\newtheorem*{remark}{Remark}
\newtheorem*{solution}{Solution}
\begin{document}
\maketitle

{
\setcounter{tocdepth}{1}
\tableofcontents
}
\hypertarget{groups}{%
\chapter{12 Groups}\label{groups}}

\begin{Shaded}
\begin{Highlighting}[]
\FunctionTok{library}\NormalTok{(datasauRus)}
\FunctionTok{head}\NormalTok{(datasauRus}\SpecialCharTok{::}\NormalTok{datasaurus\_dozen, }\AttributeTok{n=} \DecValTok{50}\NormalTok{)}
\CommentTok{\#\textgreater{}    dataset       x       y}
\CommentTok{\#\textgreater{} 1     dino 55.3846 97.1795}
\CommentTok{\#\textgreater{} 2     dino 51.5385 96.0256}
\CommentTok{\#\textgreater{} 3     dino 46.1538 94.4872}
\CommentTok{\#\textgreater{} 4     dino 42.8205 91.4103}
\CommentTok{\#\textgreater{} 5     dino 40.7692 88.3333}
\CommentTok{\#\textgreater{} 6     dino 38.7179 84.8718}
\CommentTok{\#\textgreater{} 7     dino 35.6410 79.8718}
\CommentTok{\#\textgreater{} 8     dino 33.0769 77.5641}
\CommentTok{\#\textgreater{} 9     dino 28.9744 74.4872}
\CommentTok{\#\textgreater{} 10    dino 26.1538 71.4103}
\CommentTok{\#\textgreater{} 11    dino 23.0769 66.4103}
\CommentTok{\#\textgreater{} 12    dino 22.3077 61.7949}
\CommentTok{\#\textgreater{} 13    dino 22.3077 57.1795}
\CommentTok{\#\textgreater{} 14    dino 23.3333 52.9487}
\CommentTok{\#\textgreater{} 15    dino 25.8974 51.0256}
\CommentTok{\#\textgreater{} 16    dino 29.4872 51.0256}
\CommentTok{\#\textgreater{} 17    dino 32.8205 51.0256}
\CommentTok{\#\textgreater{} 18    dino 35.3846 51.4103}
\CommentTok{\#\textgreater{} 19    dino 40.2564 51.4103}
\CommentTok{\#\textgreater{} 20    dino 44.1026 52.9487}
\CommentTok{\#\textgreater{} 21    dino 46.6667 54.1026}
\CommentTok{\#\textgreater{} 22    dino 50.0000 55.2564}
\CommentTok{\#\textgreater{} 23    dino 53.0769 55.6410}
\CommentTok{\#\textgreater{} 24    dino 56.6667 56.0256}
\CommentTok{\#\textgreater{} 25    dino 59.2308 57.9487}
\CommentTok{\#\textgreater{} 26    dino 61.2821 62.1795}
\CommentTok{\#\textgreater{} 27    dino 61.5385 66.4103}
\CommentTok{\#\textgreater{} 28    dino 61.7949 69.1026}
\CommentTok{\#\textgreater{} 29    dino 57.4359 55.2564}
\CommentTok{\#\textgreater{} 30    dino 54.8718 49.8718}
\CommentTok{\#\textgreater{} 31    dino 52.5641 46.0256}
\CommentTok{\#\textgreater{} 32    dino 48.2051 38.3333}
\CommentTok{\#\textgreater{} 33    dino 49.4872 42.1795}
\CommentTok{\#\textgreater{} 34    dino 51.0256 44.1026}
\CommentTok{\#\textgreater{} 35    dino 45.3846 36.4103}
\CommentTok{\#\textgreater{} 36    dino 42.8205 32.5641}
\CommentTok{\#\textgreater{} 37    dino 38.7179 31.4103}
\CommentTok{\#\textgreater{} 38    dino 35.1282 30.2564}
\CommentTok{\#\textgreater{} 39    dino 32.5641 32.1795}
\CommentTok{\#\textgreater{} 40    dino 30.0000 36.7949}
\CommentTok{\#\textgreater{} 41    dino 33.5897 41.4103}
\CommentTok{\#\textgreater{} 42    dino 36.6667 45.6410}
\CommentTok{\#\textgreater{} 43    dino 38.2051 49.1026}
\CommentTok{\#\textgreater{} 44    dino 29.7436 36.0256}
\CommentTok{\#\textgreater{} 45    dino 29.7436 32.1795}
\CommentTok{\#\textgreater{} 46    dino 30.0000 29.1026}
\CommentTok{\#\textgreater{} 47    dino 32.0513 26.7949}
\CommentTok{\#\textgreater{} 48    dino 35.8974 25.2564}
\CommentTok{\#\textgreater{} 49    dino 41.0256 25.2564}
\CommentTok{\#\textgreater{} 50    dino 44.1026 25.6410}
\FunctionTok{tail}\NormalTok{(datasauRus}\SpecialCharTok{::}\NormalTok{datasaurus\_dozen, }\AttributeTok{n=} \DecValTok{50}\NormalTok{)}
\CommentTok{\#\textgreater{}         dataset        x          y}
\CommentTok{\#\textgreater{} 1797 wide\_lines 77.06711 51.4869182}
\CommentTok{\#\textgreater{} 1798 wide\_lines 75.01714 46.6224426}
\CommentTok{\#\textgreater{} 1799 wide\_lines 76.66531 38.4402510}
\CommentTok{\#\textgreater{} 1800 wide\_lines 77.91587 45.9268434}
\CommentTok{\#\textgreater{} 1801 wide\_lines 73.74205 39.1209853}
\CommentTok{\#\textgreater{} 1802 wide\_lines 75.32982 32.8303519}
\CommentTok{\#\textgreater{} 1803 wide\_lines 63.41044 38.3777356}
\CommentTok{\#\textgreater{} 1804 wide\_lines 68.85649 43.0841472}
\CommentTok{\#\textgreater{} 1805 wide\_lines 66.33779 33.3065100}
\CommentTok{\#\textgreater{} 1806 wide\_lines 64.20372 26.6441143}
\CommentTok{\#\textgreater{} 1807 wide\_lines 64.49863 22.8635013}
\CommentTok{\#\textgreater{} 1808 wide\_lines 68.89099 27.2962057}
\CommentTok{\#\textgreater{} 1809 wide\_lines 72.37152 21.9616397}
\CommentTok{\#\textgreater{} 1810 wide\_lines 69.76542 19.9998505}
\CommentTok{\#\textgreater{} 1811 wide\_lines 68.62131 18.9156764}
\CommentTok{\#\textgreater{} 1812 wide\_lines 64.29774 20.4287497}
\CommentTok{\#\textgreater{} 1813 wide\_lines 66.69927 18.5910853}
\CommentTok{\#\textgreater{} 1814 wide\_lines 67.54453 16.4479381}
\CommentTok{\#\textgreater{} 1815 wide\_lines 63.94695 18.6928454}
\CommentTok{\#\textgreater{} 1816 wide\_lines 64.38819 15.7728123}
\CommentTok{\#\textgreater{} 1817 wide\_lines 65.57005 23.7657582}
\CommentTok{\#\textgreater{} 1818 wide\_lines 38.40284 19.0468587}
\CommentTok{\#\textgreater{} 1819 wide\_lines 37.83236 14.4694894}
\CommentTok{\#\textgreater{} 1820 wide\_lines 36.90416 13.5838158}
\CommentTok{\#\textgreater{} 1821 wide\_lines 36.28614 17.1057707}
\CommentTok{\#\textgreater{} 1822 wide\_lines 62.78663 13.9189931}
\CommentTok{\#\textgreater{} 1823 wide\_lines 66.81768 11.4124972}
\CommentTok{\#\textgreater{} 1824 wide\_lines 66.75502 18.0853051}
\CommentTok{\#\textgreater{} 1825 wide\_lines 65.41553 10.4635122}
\CommentTok{\#\textgreater{} 1826 wide\_lines 36.94633 13.5143775}
\CommentTok{\#\textgreater{} 1827 wide\_lines 37.82543  9.6010343}
\CommentTok{\#\textgreater{} 1828 wide\_lines 36.72284  9.3333021}
\CommentTok{\#\textgreater{} 1829 wide\_lines 67.07332  6.0492146}
\CommentTok{\#\textgreater{} 1830 wide\_lines 64.60182 12.0019170}
\CommentTok{\#\textgreater{} 1831 wide\_lines 65.43728 15.5453861}
\CommentTok{\#\textgreater{} 1832 wide\_lines 67.00402 15.3458266}
\CommentTok{\#\textgreater{} 1833 wide\_lines 66.72419  5.2498055}
\CommentTok{\#\textgreater{} 1834 wide\_lines 68.30762 13.2809165}
\CommentTok{\#\textgreater{} 1835 wide\_lines 68.76805 13.5214566}
\CommentTok{\#\textgreater{} 1836 wide\_lines 74.16727  5.3498809}
\CommentTok{\#\textgreater{} 1837 wide\_lines 64.90036 16.2452584}
\CommentTok{\#\textgreater{} 1838 wide\_lines 68.76343  8.7005729}
\CommentTok{\#\textgreater{} 1839 wide\_lines 66.81691 12.2732943}
\CommentTok{\#\textgreater{} 1840 wide\_lines 67.30935  0.2170063}
\CommentTok{\#\textgreater{} 1841 wide\_lines 34.73183 19.6017951}
\CommentTok{\#\textgreater{} 1842 wide\_lines 33.67444 26.0904902}
\CommentTok{\#\textgreater{} 1843 wide\_lines 75.62726 37.1287519}
\CommentTok{\#\textgreater{} 1844 wide\_lines 40.61013 89.1362399}
\CommentTok{\#\textgreater{} 1845 wide\_lines 39.11437 96.4817513}
\CommentTok{\#\textgreater{} 1846 wide\_lines 34.58383 89.5889020}
\end{Highlighting}
\end{Shaded}

\hypertarget{hello-bookdown}{%
\chapter{Hello bookdown}\label{hello-bookdown}}

All chapters start with a first-level heading followed by your chapter title, like the line above. There should be only one first-level heading (\texttt{\#}) per .Rmd file.

\hypertarget{a-section}{%
\section{A section}\label{a-section}}

All chapter sections start with a second-level (\texttt{\#\#}) or higher heading followed by your section title, like the sections above and below here. You can have as many as you want within a chapter.

\hypertarget{an-unnumbered-section}{%
\subsection*{An unnumbered section}\label{an-unnumbered-section}}
\addcontentsline{toc}{subsection}{An unnumbered section}

Chapters and sections are numbered by default. To un-number a heading, add a \texttt{\{.unnumbered\}} or the shorter \texttt{\{-\}} at the end of the heading, like in this section.

\hypertarget{parts}{%
\chapter{Parts}\label{parts}}

You can add parts to organize one or more book chapters together. Parts can be inserted at the top of an .Rmd file, before the first-level chapter heading in that same file.

Add a numbered part: \texttt{\#\ (PART)\ Act\ one\ \{-\}} (followed by \texttt{\#\ A\ chapter})

Add an unnumbered part: \texttt{\#\ (PART\textbackslash{}*)\ Act\ one\ \{-\}} (followed by \texttt{\#\ A\ chapter})

Add an appendix as a special kind of un-numbered part: \texttt{\#\ (APPENDIX)\ Other\ stuff\ \{-\}} (followed by \texttt{\#\ A\ chapter}). Chapters in an appendix are prepended with letters instead of numbers.

\hypertarget{footnotes-and-citations}{%
\chapter{Footnotes and citations}\label{footnotes-and-citations}}

\hypertarget{footnotes}{%
\section{Footnotes}\label{footnotes}}

Footnotes are put inside the square brackets after a caret \texttt{\^{}{[}{]}}. Like this one \footnote{This is a footnote.}.

\hypertarget{citations}{%
\section{Citations}\label{citations}}

Reference items in your bibliography file(s) using \texttt{@key}.

For example, we are using the \textbf{bookdown} package \citep{R-bookdown} (check out the last code chunk in index.Rmd to see how this citation key was added) in this sample book, which was built on top of R Markdown and \textbf{knitr} \citep{xie2015} (this citation was added manually in an external file book.bib).
Note that the \texttt{.bib} files need to be listed in the index.Rmd with the YAML \texttt{bibliography} key.

The \texttt{bs4\_book} theme makes footnotes appear inline when you click on them. In this example book, we added \texttt{csl:\ chicago-fullnote-bibliography.csl} to the \texttt{index.Rmd} YAML, and include the \texttt{.csl} file. To download a new style, we recommend: \url{https://www.zotero.org/styles/}

The RStudio Visual Markdown Editor can also make it easier to insert citations: \url{https://rstudio.github.io/visual-markdown-editing/\#/citations}

\hypertarget{blocks}{%
\chapter{Blocks}\label{blocks}}

\hypertarget{equations}{%
\section{Equations}\label{equations}}

Here is an equation.

\begin{equation} 
  f\left(k\right) = \binom{n}{k} p^k\left(1-p\right)^{n-k}
  \label{eq:binom}
\end{equation}

You may refer to using \texttt{\textbackslash{}@ref(eq:binom)}, like see Equation \eqref{eq:binom}.

\hypertarget{theorems-and-proofs}{%
\section{Theorems and proofs}\label{theorems-and-proofs}}

Labeled theorems can be referenced in text using \texttt{\textbackslash{}@ref(thm:tri)}, for example, check out this smart theorem \ref{thm:tri}.

\begin{theorem}
\protect\hypertarget{thm:tri}{}\label{thm:tri}For a right triangle, if \(c\) denotes the \emph{length} of the hypotenuse
and \(a\) and \(b\) denote the lengths of the \textbf{other} two sides, we have
\[a^2 + b^2 = c^2\]
\end{theorem}

Read more here \url{https://bookdown.org/yihui/bookdown/markdown-extensions-by-bookdown.html}.

\hypertarget{callout-blocks}{%
\section{Callout blocks}\label{callout-blocks}}

The \texttt{bs4\_book} theme also includes special callout blocks, like this \texttt{.rmdnote}.

You can use \textbf{markdown} inside a block.

\begin{Shaded}
\begin{Highlighting}[]
\FunctionTok{head}\NormalTok{(beaver1, }\AttributeTok{n =} \DecValTok{5}\NormalTok{)}
\CommentTok{\#\textgreater{}   day time  temp activ}
\CommentTok{\#\textgreater{} 1 346  840 36.33     0}
\CommentTok{\#\textgreater{} 2 346  850 36.34     0}
\CommentTok{\#\textgreater{} 3 346  900 36.35     0}
\CommentTok{\#\textgreater{} 4 346  910 36.42     0}
\CommentTok{\#\textgreater{} 5 346  920 36.55     0}
\end{Highlighting}
\end{Shaded}

It is up to the user to define the appearance of these blocks for LaTeX output.

You may also use: \texttt{.rmdcaution}, \texttt{.rmdimportant}, \texttt{.rmdtip}, or \texttt{.rmdwarning} as the block name.

The R Markdown Cookbook provides more help on how to use custom blocks to design your own callouts: \url{https://bookdown.org/yihui/rmarkdown-cookbook/custom-blocks.html}

\hypertarget{sharing-your-book}{%
\chapter{Sharing your book}\label{sharing-your-book}}

\hypertarget{publishing}{%
\section{Publishing}\label{publishing}}

HTML books can be published online, see: \url{https://bookdown.org/yihui/bookdown/publishing.html}

\hypertarget{pages}{%
\section{404 pages}\label{pages}}

By default, users will be directed to a 404 page if they try to access a webpage that cannot be found. If you'd like to customize your 404 page instead of using the default, you may add either a \texttt{\_404.Rmd} or \texttt{\_404.md} file to your project root and use code and/or Markdown syntax.

\hypertarget{metadata-for-sharing}{%
\section{Metadata for sharing}\label{metadata-for-sharing}}

Bookdown HTML books will provide HTML metadata for social sharing on platforms like Twitter, Facebook, and LinkedIn, using information you provide in the \texttt{index.Rmd} YAML. To setup, set the \texttt{url} for your book and the path to your \texttt{cover-image} file. Your book's \texttt{title} and \texttt{description} are also used.

This \texttt{bs4\_book} provides enhanced metadata for social sharing, so that each chapter shared will have a unique description, auto-generated based on the content.

Specify your book's source repository on GitHub as the \texttt{repo} in the \texttt{\_output.yml} file, which allows users to view each chapter's source file or suggest an edit. Read more about the features of this output format here:

\url{https://pkgs.rstudio.com/bookdown/reference/bs4_book.html}

Or use:

\begin{Shaded}
\begin{Highlighting}[]
\NormalTok{?bookdown}\SpecialCharTok{::}\NormalTok{bs4\_book}
\end{Highlighting}
\end{Shaded}


  \bibliography{book.bib,packages.bib}

\end{document}
